% !TEX root = main.tex
\chapter{Einleitung}

Der Large Hadron Collider (\lhc) ist der aktuell größte und leistungsstärkste Teilchenbeschleuniger der Welt. Er wird am Europäischen Kernforschungszentrum \cern bei Genf in der Schweiz betrieben. Am \lhc sind vier große Experimente platziert. Eines dieser vier großen Experimente ist das \lhcb-Experiment, welches in Präzisionsmessungen im Flavour-Sektor nach indirekten Anzeichen Neuer Physik sucht. Neue Physik beschreibt dabei Prozesse, die nicht durch das Standardmodell der Teilchenphysik (SM) beschrieben werden.\\
\\
Das \lhcb- (Large Hadron Collider beauty) Experiment ist darauf spezialisiert Prozesse mit Hadronen, die \bquark- oder \cquark-Quarks enthalten, zu messen. Im Vordergrund stehen dabei Bestimmungen der Zerfallsbreiten seltener Zerfälle oder präzise Vermessungen von \CP-verletzenden Parametern des SM.\\
\\
Für die möglichst exakte Messung der \CP-verletzenden Parameter des SM werden dabei zeitaufgelöste Messungen durchgeführt. Dafür ist es notwendig, den initialen Flavour der zerfallenden \B-Mesonen zu kennen; also ob das \B-Meson bei der Produktion ein \bquark oder ein \bquarkbar-Quark enthielt. Diese Information wird bei \lhcb durch das Flavour Tagging gewonnen. Verschiedene Algorithmen des Flavour Taggings geben dafür zu jedem Ereignis eine Entscheidung (tag) um welches \B-Meson es sich initial gehandelt hat und eine Selbsteinschätzung (mistag) aus, der beschreibt wie groß die Wahrscheinlichkeit ist, mit dieser Entscheidung falsch zu liegen. Da der tag und der mistag direkt in die Analyse der \CP-verletzenden Parameter eingehen ist es wichtig das Flavour Tagging möglichst exakt zu kalibrieren. Um dies auf Daten zu tun, sind selbsttaggende Zerfälle nötig, also Zerfälle, bei denen aus dem Endzustand auf den Zerfallsflavour des \B-Mesons geschlossen werden kann. \\
\\
Um die Qualität des Flavour Taggings zu steigern sind zum einen neue Algorithmen (Tagger) entwickelt worden \cite{charm-tagger} und zum anderen werden die Ausgaben verschiedener Tagger kombiniert. Bei dieser Kombination sollten die einzelnen Tagging Algorithmen möglichst gering korreliert sein, um durch Kombination die Wahrscheinlichkeit für falsche Tags nicht zu unterschätzen.\\
\\
In dieser Arbeit wird die Kalibration verschiedene Tagger auf dem Kanal \mbox{\BdToDpi} überprüft. Außerdem werden zwei auf diesem Kanal neu entwickelte Tagger untersucht, und mögliche Korrelationen zwischen Taggern abgeschätzt, die ähnliche Prozesse betrachten. 
