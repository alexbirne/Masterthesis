% !TEX root = main.tex
\chapter{Zusammenfassung und Ausblick}

Im Zuge der vorgestellten Arbeit wurden im Bereich des Flavour Taggings am \lhcb-Experiment verschiedene Tagger untersucht. Dabei wurden zunächst auf der Opposite Side für die OS Kombination, den OS Charm Tagger und den OS Kaon nnet Tagger die Kalibrierung der mistag-Wahrscheinlichkeiten $\eta$ in Bezug auf die wahren mistag-Wahrscheinlichkeiten $\omega$ überprüft.\\
Weiter wurde dieser Zusammenhang auch auf der Same Side für drei Tagger überprüft. Der SS Pion Tagger wies dabei deutlich stärkere Abweichungen von einer idealen Kalibration auf als die Tagger der Opposite Side. Für die beiden weiteren Tagger, den SS Proton und den SS Pion BDT Tagger, zeigten sich auch Abweichungen vom ideal kalibrierten Fall. Allerdings lag der Fokus hier weniger auf der Gegenprobe der Kalibrierung, sondern vielmehr auf der grundsätzlichen Funktionalität beider Tagger, da es sich um Neuentwicklungen handelt. Nach einigen Problemen bei der Entwicklung, konnte in dieser Arbeit diese Funktionalität das erste Mal bestätigt werden. Außerdem ließ sich hier die effektive Taggingeffizienz $\varepsilon D^2$ des SS Pion Taggers und  seines zukünftigen Nachfolgers, dem SS Pion BDT Tagger, vergleichen. Dabei ergab sich für den SS Pion BDT eine etwa \SI{0{,}6}{\%} höhere effektive Taggingeffizienz.\\
Da sowohl der SS Proton als auch die beiden Pionen Tagger den Hadronisierungsprozess des gleichen Quarks untersuchen, wurden hier außerdem erste Untersuchungen einer möglichen Korrelation der verschiedenen Tagger auf der Same Side untersucht. Eine quantitative Aussage ist an dieser Stelle noch nicht möglich. Die beobachteten Korrelation zwischen den mistag-Verteilungen und den Taggingentscheidungen sind jedoch größer zwischen dem SS Proton und dem SS Pion BDT Tagger als zwischen dem SS Proton und dem SS Pion Tagger. \\ 
\\
Weiterhin wurde bei der Kalibrierung der Tagger die Mischungsfrequenz $\dmd$  der neutralen \B-Mesonen bestimmt. Diese Messung wurde blind durchgeführt. Es konnte jedoch gezeigt werden, dass der statistische Fehler auf diese Observable mit dem \num{2012} bei \lhcb aufgenommenen Datensatz nur bei Verwendung der durch die OS Kombination getaggten Ereignisse bereits kleiner ist als bei einer älteren Analyse auf dem \num{2011} aufgenommenen Datensatz, bei der eine Kombination aus der OS Kombination und dem SS Pion Tagger verwendet wurde. Außerdem zeigt die neue Selektion auf den Datensätzen beider Jahre eine bessere Reinheit. \\
\\
Zur Kalibrierung des Flavour Taggings wird aktuell eine Methodik angewendet, bei der die mistag-Wahrscheinlichkeiten $\eta$ auf true-mistag-Wahrscheinlichkeiten $\omega$ kalibriert werden. Hier wurde eine alternative Herangehensweise vorgestellt, bei der zwischen \Bz- und \Bzb-Mesonen unterschieden werden kann. Dabei werden die mistag-Wahrscheinlichkeiten $\eta$ in anschaulichere $P(\bquarkbar)$, die Wahrscheinlichkeit, dass es sich bei einem Teilchen um ein \Bz-Meson handelt, umgerechnet. Dieses Verfahren bietet sich auch in hinsichtlich dessen an, dass die Tagger selbst intern mit der Wahrscheinlichkeit $P(\bquarkbar)$ rechnen. Da auf Daten jedoch nur über den Tag $d$ zwischen \Bz- und \Bzb-Meson unterschieden werden kann, und der \enquote{wahre} Flavour nicht bekannt ist, sind die Parameter $\Delta p_0$ und $\Delta p_1$, die eine mögliche Taggingasymmetrie berücksichtigen, in diesem Verfahren zunächst nicht korrekt bestimmbar.\\
\\
Die Ergebnisse der Prüfung der Kalibrierung der verschiedenen Tagger auf dem Kanal \BdToDpi werden in die Bestimmung der Systematiken der Standard Kalibration der Flavour Tagging Gruppe eingehen. Daneben müssen für die hier vorgestellten Parameter allerdings auch noch die systematische Unsicherheiten bestimmt werden. Außerdem sind weitere Proben auf einem kombinierten Datensatz aus den Jahren \num{2011} und \num{2012} möglich. 